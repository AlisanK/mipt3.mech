\documentclass{weekly}
\begin{document}
\maketitlew{Аналитическая механика}{1}{5}{19}

\paragraph{5.9.} Парашютист массы~$m$ прыгает с~самолёта, летящего
горизонтально на~высоте~$H$ со~скоростью~$v_0$. По~какой траектории
движется парашютист при~затяжном прыжке (до~момента раскрытия
парашюта), если сила сопротивления воздуха~$\vec F = -\beta \vec v$,
где~$\vec v$~--- скорость парашютиста, а~изменение ускорения свободного
падения с~высотой не~учитывается? Из~полученного уравнения
предельным переходом~$\beta \to 0$ найти уравнение траектории
в~отсутствие сил сопротивления.

$\blacktriangleright$ Введём координатную систему:
ось~$x$ направим по~$\vec v_0$ (горизонтально),
ось~$y$~--- вертикально вверх,
точке прыжка сопоставим координаты~$(0; H)$.
Запишем уравнения движения парашютиста,
рассматривая его как~материальную точку (II~закон Ньютона):
\begin{equation}
    m\ddot{\vec r} = -\beta\dot{\vec r} + m\vec g
\quad \Longleftrightarrow \quad
    \begin{cases}
        \ddot x + \frac{\beta}{m}\dot x = 0; \\
        \ddot y + \frac{\beta}{m}\dot y = -g.
    \end{cases}
    \label{5.9:Newt}
\end{equation}
Интегрируя уравнения~\eqref{5.9:Newt} в~координатах, имеем
\begin{equation}
\begin{cases}
    x(t) = C_{x,1} \exp\left( -\frac{\beta}{m} t \right) + C_{x,2}; \\
    y(t) = C_{y,1} \exp\left( -\frac{\beta}{m} t \right) + C_{y,2}
            - \frac{mg}{\beta} t.
    \label{5.9:xy-gen}
\end{cases}
\end{equation}
Начальные условия задачи Коши:
\begin{equation}
\begin{cases}
    x(0) = 0, & \dot x(0) = v_0; \\
    y(0) = H, & \dot y(0) = 0.
    \label{5.9:xy-init}
\end{cases}
\end{equation}
Из~\eqref{5.9:xy-gen} и~\eqref{5.9:xy-init} следует параметрический
(по~времени) вид траектории:
\begin{equation}
\begin{cases}
    x(t) = \frac{mv_0}{\beta}
            \left[ 1 - \exp\left( -\frac{\beta}{m} t \right) \right]; \\
    y(t) = H + \frac{m^2 g}{\beta^2}
            \left[ 1 - \exp\left( -\frac{\beta}{m} t \right) \right] -
            \frac{mg}{\beta} t.
\end{cases}
\end{equation}

Выразим~$\left[ 1 - \exp\left( -\frac{\beta}{m} t \right) \right]$
и~$t$ через~$x(t)$ и~подставим в~$y(t)$:
\begin{gather}
    1 - \exp\left( -\frac{\beta}{m} t \right)
        = \frac{\beta x}{m v_0}; \\
    t = -\frac{m}{\beta} \ln\left(1 - \frac{\beta x}{m v_0}\right); \\
    y(x) = H + \frac{mg}{\beta v_0} x +
            \frac{m^2 g}{\beta^2}
            \ln\left( 1 - \frac{\beta x}{m v_0}\right).
\end{gather}

Устремив~$\beta \to 0$, разложим последнее слагаемое
выражения зависимости~$y(x)$ в~ряд Маклорена по~$\beta$ до~$o(\beta^3)$:
\begin{equation}
    y(x)\big|_{\beta \to 0} = H + \cancel{\frac{mg}{\beta v_0} x -
            \frac{mg}{\beta v_0}x} - \frac{g}{2v_0^2}x^2 + o(\beta^3).
\end{equation}

\textbf{Ответ:}\quad
$y(x) = H + \dfrac{mg}{\beta v_0} x + \dfrac{m^2 g}{\beta^2}
\ln\left( 1 - \dfrac{\beta x}{m v_0}\right)
\stackrel{\beta \to 0}{\simeq} H - \dfrac{g}{2v_0^2}x^2$.
\hfill $\blacktriangleleft$


\paragraph{6.18.} Однородный диск массы~$m$ может катиться без~скольжения
по~горизонтальной прямой. К~центру диска прикладывается горизонтальная
сила, в~результате чего центр диска начинает колебаться
по~синусоидальному закону с~амплитудой~$a$ и~частотой~$\omega$.
Найти зависимость силы трения от~времени.

$\blacktriangleright$ К~диску приложены три~силы:
к~центру диска~--- сила тяжести~$m\vec g$,
в~точке касания диска и~прямой~--- сила нормальной реакции
и~сила трения~$\vec f$.
Направим ось~$x$ параллельно прямой. Поскольку скорость точки касания
направлена вдоль прямой, $\vec f(t) = f(t) \hat x$.

Условие движения диска без~проскальзывания приводит к~соотношению
\begin{equation}
    \ddot x = R \ddot\varphi.
\end{equation}
Поскольку $x(t) = a\sin\omega t$ (с~точностью до~выбора начального
момента времени), уравнение вращательного движения диска запишется в~виде
\begin{equation}
    \frac12 mR^2 \ddot\varphi(t) = f(t) R,
\end{equation}
откуда
\begin{equation}
    f(t) = -\frac12 ma\omega^2 \sin\omega t.
\end{equation}

\textbf{Ответ:}\quad зависимость силы трения от~времени
имеет синусоидальный вид с~частотой~$\omega$
и~амплитудой~$f_{\max} = ma\omega^2/2$.
\hfill $\blacktriangleleft$


\paragraph{6.30.} В~упражнении с~обручем гимнастка сообщает центру
однородного кругового обруча радиуса~$r$ горизонтальную скорость~$v_0$
и~закручивает его с~угловой скоростью~$\omega_0$.
Коэффициент трения между обручем и~полом равен~$f$
(во~время движения обруч не~подпрыгивает).
Как~должны быть связаны величины~$v_0$ и~$\omega_0$ для того,
чтобы обруч вернулся в~исходное положения за~время~$t$,
определяемое музыкальным сопровождением?

$\blacktriangleright$ Для~простоты восприятия решения
условимся считать положительным направление угловой скорости
\emph{против~часовой стрелки,} вдоль прямой~--- \emph{вправо.}

В~режиме проскальзывания зависимости скорости и~ускорения от~времени
имеют вид
\begin{align}
    v &= v_0 - fgt; \\
    \omega &= \omega_0 - \frac{fgt}{r}.
\end{align}
Проскальзывание прекращается, когда~$v = -\omega r$,
т.\,е. за~время
\begin{equation}
    t_0 = \frac{v_0 + \omega_0 r}{2fg}.
\end{equation}
В~случае, если~$t \leqslant t_0$ (проскальзывание не~прекращается
до~возврата в~начальную точку), достаточно потребовать
\begin{align}
    0 = v_0 t - \frac12 fgt^2
&\then
    v_0 = \frac12 fgt;
\\
    \frac{v_0 + \omega_0 r}{2fg} \geqslant t
            = \frac{2v_0}{fg}
&\then
    3v_0 \leqslant \omega_0 r.
\end{align}

В~противном случае в~момент окончания проскальзывания скорость
обруча составит
\begin{equation}
    v_1 = \frac{v_0 - \omega_0 r}{2} \leqslant 0,
\end{equation}
откуда, кстати, следует ограничение $v_0 \leqslant \omega_0 r$,
а~оставшийся до~возврата путь
\begin{equation}
    S = v_0 t_0 - \frac12 fg t_0^2
        = \frac{v_0 + \omega_0 r}{2fg} 
            \left(v_0 - \frac{v_0 + \omega_0 r}{4}\right).
\end{equation}

Тогда полное время движения
\begin{gather}
    t = t_0 + \frac{S}{\abs{v_1}}
        = \frac{v_0 + \omega_0 r}{2fg} +
            \frac{v_0 + \omega_0 r}{2fg}
            \left(v_0 - \frac{v_0 + \omega_0 r}{4}\right)
            \frac{2}{\omega_0 r - v_0},
    \label{6.30:gibberish}
\end{gather}
что и~является искомым уравнением, связывающим~$v_0$
и~$\omega_0$ в~этом случае.

\medskip
\textbf{Ответ:}\quad
$\begin{cases}
    v_0 = fgt/2,
        & \omega_0 r \geqslant 3v_0; \\
    \text{ур-ние~\eqref{6.30:gibberish}},
        & v_0 \leqslant \omega_0 r \leqslant 3v_0.
\end{cases}$
\hfill $\blacktriangleleft$


\bigskip
\paragraph{6.34.} Пластина массы~$m$ может двигаться в~неподвижной
плоскости~$xy$. Положение пластины задаётся координатами~$x$, $y$
полюса~$P$ и~углом~$\varphi$, который прямая, соединяющая полюс~$P$
с~центром масс пластины~$C$, образует с~осью~$Ox$.
Составить уравнения плоскопараллельного движения пластины
в~переменных~$x, y, \varphi$, если момент инерции пластины
относительно оси, проходящей через полюс~$P$ перпендикулярно
плоскости пластины, равен~$J$, а~$PC = l$.

$\blacktriangleright$ По~теореме о~движении центра масс
($\vec R$~--- главный вектор сил, приложенных к~пластине)
\begin{equation}
    \vec R = \cvec{R_x}{R_y}{0}
        = m\ddot{\vec r}_C
        = m \dd{{}^2}{t^2} \cvec{x+l\cos\varphi}{y+l\sin\varphi}{0}
        = m \cvec{\ddot x - l\dot\varphi^2\cos\varphi -
                l\sin\varphi \ddot\varphi}
            {\ddot y - l\dot\varphi^2\sin\varphi +
                l\cos\varphi \ddot\varphi}{0}.
\end{equation}

По~определению углового момента
\begin{equation}
    L_z = (J - ml^2)\dot\varphi +
            ml (v_{C,y} \cos\varphi - v_{C,x} \sin\varphi).
\end{equation}
Оставшееся соотношение нетрудно продифференцировать,
при~этом~$\dot L_z = M_z$~--- модуль главного вектора
момента внешних сил.

\bigskip
\textbf{Ответ:}\quad
$\begin{cases}
    R_x = m\ddot x - ml\dot\varphi^2 \cos\varphi -
            ml\ddot\varphi\sin\varphi; \\
    R_y = m\ddot y - ml\dot\varphi^2 \sin\varphi +
            ml\ddot\varphi\cos\varphi; \\
    M_z = J\ddot\varphi +
            ml(\ddot y \cos\varphi - \ddot x \sin\varphi).
\end{cases}$
\hfill $\blacktriangleleft$


\paragraph{7.11.} Показать, что~потенциальная энергия пружины,
состоящей из~двух последовательно соединённых частей~$AB$ и~$BD$
с~жёсткостями~$c_1$ и~$c_2$ соответственно, совпадает
с~потенциальной энергией пружины жёсткости
\begin{equation}
    c = \frac{c_1 c_2}{c_1 + c_2}.
\end{equation}
Решить аналогичную задачу для~параллельно соединённых пружин.

$\blacktriangleright$ В~силу равенства сил, с~которыми
части пружины действуют друг на~друга, имеем
\begin{gather}
    c_1 \Delta x_1 = c_2 \Delta x_2
\then
    \Delta x = \Delta x_1 + \Delta x_2
        = \Delta x_2 \left( 1 + \frac{c_2}{c_1} \right);
\\
    W_{посл} = \frac12 c_1 \Delta x_1^2 + \frac12 c_2 \Delta x_2^2
        = \frac{\Delta x^2}{2} \frac{c_1 c_2}{c_1 + c_2},
\end{gather}
что~и требовалось показать. \qed

Для~параллельно соединённых пружин
$\Delta x = \Delta x_1 = \Delta x_2$,
\begin{equation}
\tag*{$\blacktriangleleft$}
    W_{пар} = \frac{c_1 + c_2}{2} \Delta x^2.
\end{equation}

\end{document}
