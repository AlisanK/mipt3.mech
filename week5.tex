\documentclass{weekly}
\begin{document}
\maketitlew{Аналитическая механика}{1}{5}{19}

\paragraph{5.9.} Парашютист массы~$m$ прыгает с~самолёта, летящего
горизонтально на~высоте~$H$ со~скоростью~$v_0$. По~какой траектории
движется парашютист при~затяжном прыжке (до~момента раскрытия
парашюта), если сила сопротивления воздуха~$\vec F = -\beta \vec v$,
где~$\vec v$~--- скорость парашютиста, а~изменение ускорения свободного
падения с~высотой не~учитывается? Из~полученного уравнения
предельным переходом~$\beta \to 0$ найти уравнение траектории
в~отсутствие сил сопротивления.

$\blacktriangleright$ Введём координатную систему:
ось~$x$ направим по~$\vec v_0$ (горизонтально),
ось~$y$~--- вертикально вверх,
точке прыжка сопоставим координаты~$(0; H)$.
Запишем уравнения движения парашютиста,
рассматривая его как~материальную точку (II~закон Ньютона):
\begin{equation}
    m\ddot{\vec r} = -\beta\dot{\vec r} + m\vec g
\quad \Longleftrightarrow \quad
    \begin{cases}
        \ddot x + \frac{\beta}{m}\dot x = 0; \\
        \ddot y + \frac{\beta}{m}\dot y = -g.
    \end{cases}
    \label{5.9:Newt}
\end{equation}
Интегрируя уравнения~\eqref{5.9:Newt} в~координатах, имеем
\begin{equation}
\begin{cases}
    x(t) = C_{x,1} \exp\left( -\frac{\beta}{m} t \right) + C_{x,2}; \\
    y(t) = C_{y,1} \exp\left( -\frac{\beta}{m} t \right) + C_{y,2}
            - \frac{mg}{\beta} t.
    \label{5.9:xy-gen}
\end{cases}
\end{equation}
Начальные условия задачи Коши:
\begin{equation}
\begin{cases}
    x(0) = 0, & \dot x(0) = v_0; \\
    y(0) = H, & \dot y(0) = 0.
    \label{5.9:xy-init}
\end{cases}
\end{equation}
Из~\eqref{5.9:xy-gen} и~\eqref{5.9:xy-init} следует параметрический
(по~времени) вид траектории:
\begin{equation}
\begin{cases}
    x(t) = \frac{mv_0}{\beta}
            \left[ 1 - \exp\left( -\frac{\beta}{m} t \right) \right]; \\
    y(t) = H + \frac{m^2 g}{\beta^2}
            \left[ 1 - \exp\left( -\frac{\beta}{m} t \right) \right] -
            \frac{mg}{\beta} t.
\end{cases}
\end{equation}

Выразим~$\left[ 1 - \exp\left( -\frac{\beta}{m} t \right) \right]$
и~$t$ через~$x(t)$ и~подставим в~$y(t)$:
\begin{gather}
    1 - \exp\left( -\frac{\beta}{m} t \right)
        = \frac{\beta x}{m v_0}; \\
    t = -\frac{m}{\beta} \ln\left(1 - \frac{\beta x}{m v_0}\right); \\
    y(x) = H + \frac{mg}{\beta v_0} x +
            \frac{m^2 g}{\beta^2}
            \ln\left( 1 - \frac{\beta x}{m v_0}\right).
\end{gather}

Устремив~$\beta \to 0$, разложим последнее слагаемое
выражения зависимости~$y(x)$ в~ряд Маклорена по~$\beta$ до~$o(\beta^3)$:
\begin{equation}
    y(x)\big|_{\beta \to 0} = H + \cancel{\frac{mg}{\beta v_0} x -
            \frac{mg}{\beta v_0}x} - \frac{g}{2v_0^2}x^2 + o(\beta^3).
\end{equation}

\textbf{Ответ:}\quad
$y(x) = H + \dfrac{mg}{\beta v_0} x + \dfrac{m^2 g}{\beta^2}
\ln\left( 1 - \dfrac{\beta x}{m v_0}\right)
\stackrel{\beta \to 0}{\simeq} H - \dfrac{g}{2v_0^2}x^2$.
\hfill $\blacktriangleleft$

\end{document}
