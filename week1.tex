\documentclass{weekly}
\begin{document}
\maketitlew{Аналитическая механика}{1}{1}{1}

\paragraph{1.20.} В~некотором приближении орбиту Меркурия
можно представить плоской розеткой, уравнение которой
в~полярных координатах имеет вид
\begin{equation}\label{1.20:tr}
    r = \frac{p}{1+e\cos\omega\varphi}, \quad
    \omega = \const \neq 1.
\end{equation}
Используя закон площадей $r^2 \dot\varphi = c = \const$,
найти зависимость ускорения~$w$ планеты от~$r$.

$\blacktriangleright$ Компоненты скорости
в~полярной системе координат известны:
\begin{align}
    v_r &= H_r \dot r = \dot r, \\
    v_\varphi &= H_\varphi \dot\varphi = r \dot\varphi; \\
    v^2 &= v_r^2 + v_\varphi^2 = \dot r^2 + r^2 \dot\varphi^2.
\end{align}
Найдём компоненты разложения ускорения
$\vec w = \vec w_r + \vec w_\varphi$:
\begin{align}
    w_r &= \frac{1}{2H_r}
            \left[ \dd{}{t} \left(\pdd{v^2}{\dot r}\right) -
            \pdd{v^2}{r} \right]
        = \dd{\dot r}{t} - r \dot\varphi^2
        = \ddot r - r\dot\varphi^2; \label{1.20:wr}\\
    w_\varphi &= \frac{1}{2H_\varphi}
            \left[ \dd{}{t} \left(\pdd{v^2}{\dot\varphi}\right) -
            \pdd{v^2}{\varphi} \right]
        = \frac{1}{r} \dd{(r^2 \dot\varphi)}{t}
        = 2 \dot r \dot\varphi + r \ddot\varphi. \label{1.20:wphi}
\end{align}
Продифференцируем по~времени уравнение~\eqref{1.20:tr} с~учётом
$\dot\varphi = \dfrac{c}{r^2}$ и
$e \cos\omega\varphi = \dfrac{p}{r} - 1$:
\begin{align}
    \dot r &= \frac{pe\omega \sin\omega\varphi \cdot \dot\varphi}
        {(1+e\cos\omega\varphi)^2}
        = \frac{\cancel{r^2}}{p} e\omega \sin\omega\varphi \cdot
            \frac{c}{\cancel{r^2}}
        = \frac{ec\omega}{p} \sin{\omega\varphi}, \\
    \ddot r &= \frac{ec\omega^2}{p} \cos\omega\varphi \cdot
        \frac{c}{r^2} = \frac{c^2 \omega^2}{pr^2}
        \left( \frac{p}{r} - 1 \right).
\end{align}
Совершив подстановку в~\eqref{1.20:wr}, получим
\begin{align}
    w_r = \frac{c^2 \omega^2}{pr^2} \left(\frac{p}{r}-1\right) -
            \frac{c^2}{r^3}
        = \frac{c^2}{pr^3} \left( p \omega^2 - r \omega^2 - p \right)
        = -\frac{c^2}{pr^3} \left[ \omega^2 r +
            \left( 1-\omega^2 \right) p \right].
\end{align}
\textsl{Примечание.} При $\omega \to 1$ приходим к~классическому
ньютоновскому закону $\omega_r = -c^2/(pr^2)$.
Осталось убедиться, что, как и~в~классическом случае, $w_\varphi = 0$.
Это видно из выражения~\eqref{1.20:wphi} и~соотношения
\begin{align}
    r \ddot\varphi &= r \cdot \dd{}{t} \left( \frac{c}{r^2} \right)
        = -\frac{2c}{r^2} \cdot \dot r = -2 \dot r \dot\varphi.
\end{align}

\textbf{Ответ:}\quad $w = \abs{w_r} = \dfrac{c^2}{pr^3} \cdot
\left.\Big| \omega^2 r + \left( 1-\omega^2 \right) p \Big|\right.$.
\hfill $\blacktriangleleft$

\end{document}
