\documentclass{weekly}
\begin{document}
\maketitlew{Аналитическая механика}{1}{7}{27}

\paragraph{8.8.} Спутник движется вокруг Земли по~эллиптической орбите
с~эксцентриситетом~$e$. Найти отношение максимального и~минимального
значений угловой скорости радиуса"=вектора спутника.

$\blacktriangleright$ ЗСМИ гласит:
\begin{equation}
    \omega r^2 = \const.
\end{equation}
\textbf{Ответ:}\qquad $\dfrac{\omega_{\max}}{\omega_{\min}} =
\dfrac{r_{\max}^2}{r_{\min}^2} = \left(\dfrac{1+e}{1-e}\right)^2$.
\hfill $\blacktriangleleft$


\paragraph{8.22.} Комета массы~$m$ движется в~поле тяготения звезды~$S$
массы~$M$, имея невозмущённую скорость~$v_\infty$ (на~бесконечности)
и~прицельное расстояние~$d$. Найти уравнение траектории кометы
и~определить угол~$\theta$, на~который отклоняется её траектория,
когда она снова удаляется в~бесконечность.

$\blacktriangleright$
В~полярных координатах
\begin{align}
    w_r &= \ddot r - r\dot\varphi^2 = -\frac{\gamma M}{r^2};
        \label{8.22:wr} \\
    w_\varphi &= r\ddot\varphi + 2\dot r \dot\varphi = 0.
\end{align}

Из ЗСМИ следует зависимость~$\dot\varphi(r)$:
\begin{equation}
    \dot\varphi = \frac{v_\infty d}{r^2}.
\end{equation}

Введём замену:
\begin{align}
    y &\equiv \frac{1}{r}; &
    y' &= \dd{y}{\varphi} = -\frac{r'}{r^2}
        = -\frac{\dot r}{\dot\varphi r^2} = -\frac{\dot r}{v_\infty d}; &
    y'' &= -\frac{\ddot r r^2}{v_\infty^2 d^2}.
\end{align}
Уравнение~\eqref{8.22:wr} примет вид
\begin{equation}
    y'' + y = \frac{\gamma M}{v_\infty^2 d^2} \then
    y = \frac{1}{r} = \frac{\gamma M}{v_\infty^2 d^2} +
        A \cos(\varphi + \varphi_0).
\end{equation}

Осталось вспомнить (задать) начальные условия:
при~$\varphi \to 0$: $y \to 0$, $\dot r \to v_\infty$, откуда
\begin{gather}
    A = -\frac{\gamma M}{v_\infty^2 d^2 \cos\varphi_0}; \\
    y'(0) = -\frac{1}{d} = \frac{\gamma M}{v_\infty^2 d^2} \tan\varphi_0
        \then \tan\varphi_0 = -\frac{v_\infty^2 d}{\gamma M}
        \equiv \xi. \\[2ex]
    r(\varphi) = \frac{\xi d}{1 - \sqrt{1 + \xi^2}
            \cos\left(\varphi - \arctan\xi\right)}.
        \label{8.22:fin-eq}
\end{gather}

Угол~$\pi - \theta$ равен модулю (генеральной) разности корней
уравнения~$y(\varphi) = 0$:
\begin{gather}
    \cos(\varphi - \varphi_0) + \cos\varphi_0 = 0
    \then
    2 \sin\frac{\varphi}{2}
        \sin\left(\frac{\varphi}{2} - \varphi_0\right) = 0
    \then
    \pi - \theta = \abs{2\varphi_0}.
\end{gather}

\textbf{Ответ:}\qquad см.~\eqref{8.22:fin-eq}; \quad
$\tan\dfrac{\theta}{2} = \dfrac{\gamma M}{v_\infty^2 d}$.
\hfill $\blacktriangleleft$


\paragraph{8.24.} С~северного полюса Земли запускается снаряд так,
что~направление начальной скорости~$v_0$ составляет угол~$\alpha$
с~горизонтом. Какой должна быть величина~$v_0$, чтобы место
падения снаряда имело географическую широту~$\varphi$?

$\blacktriangleright$ Удельный момент импульса снаряда (в~квадрате)
\begin{equation}
    v_0^2 R^2 \cos^2\alpha = \gamma M p.
    \label{8.24:mom}
\end{equation}
Уравнение траектории~--- эллипса~--- запишется с~учётом симметрии
в~виде
\begin{equation}
    R = \frac{p}{1 - e\cos\left(
        \frac{\pi}{4} + \frac{\varphi}{2}\right)}
    \then e = \frac{1 - \frac{p}{R}}{\cos\left(
        \frac{\pi}{4} + \frac{\varphi}{2}\right)};
\end{equation}
с~другой стороны, интеграл энергии выглядит как
\begin{equation}
    \frac{v_0^2}{\gamma M}
        = \frac{2}{R} - \frac{1-e^2}{p}
        = \frac{2}{R} - \frac{1}{p} +
            \frac{\left(1 - \frac{p}{R}\right)^2}
            {p \cos^2\left(
            \frac{\pi}{4} + \frac{\varphi}{2}\right)}.
    \label{8.24:eint}
\end{equation}
\textsl{Указание.} Систему~\{\eqref{8.24:mom}; \eqref{8.24:eint}\}
нужно решить относительно~$v_0$.
\hfill $\blacktriangleleft$


\paragraph{8.50.} Спутнику, движущемуся со~скоростью~$v$ по~круговой
орбите радиуса~$R$, сообщается импульс торможения, в~результате
которого скорость изменилась на~величину~$\Delta v$.
Найти параметр~$p$, эксцентриситет~$e$ новой орбиты
и~угол~$\varphi$ между радиусом"=вектором в~точке приложения
импульса и~направлением на~перигей новой орбиты.

$\blacktriangleright$ Удельный момент импульса снаряда (в~квадрате)
\begin{gather}
\begin{split}
    \left(v_0 - \Delta v\right)^2 R^2 &= \gamma M p; \\
    v_0^2 R^2 &= \gamma M R.
\end{split}\\
    p = \frac{\left(v_0 - \Delta v\right)^2}{v_0^2} R.
\end{gather}

Точка, очевидно, стала апогеем; $\varphi = \pi$.
Найдём эксцентриситет орбиты:
\begin{equation}
    R = a(1+e) = \frac{p}{1-e}
\then
    e = \frac{\Delta v \left(2v_0 - \Delta v\right)}{v_0^2}.
\end{equation}

\textbf{Ответ:}\qquad
$p = \dfrac{\left(v_0 - \Delta v\right)^2}{v_0^2} R$;\quad
$e = \dfrac{\Delta v \left(2v_0 - \Delta v\right)}{v_0^2}$;\quad
$\varphi = \pi$.
\hfill $\blacktriangleleft$

\end{document}
