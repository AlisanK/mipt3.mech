\documentclass{weekly}
\begin{document}
\maketitlew{Аналитическая механика}{2}{9}{32}

\paragraph{11.19.} Тонкий однородный диск массы~$m$ катится
без~скольжения по~неподвижной горизонтальной плоскости,
образуя с~нею угол~$\alpha$. Центр диска~$C$ описывает
горизонтальную окружность с~постоянной линейной скоростью~$v$
Точка~$O$ пересечения оси симметрии диска с~горизонтальной
плоскостью неподвижна.

$\blacktriangleright$ Диск совершает два вращения:
вокруг своей оси~($\vec\omega$) и~вокруг вертикали~($\vec\Omega$),
проведённой через~$O$.
С~учётом условия непроскальзывания (точка касания диска
и~плоскости~$Q$) и~заданной скорости центра
\begin{gather}
    \Omega = \frac{v}{R/\tan\alpha \cdot \cos\alpha}
        = \frac{v}{R} \frac{\sin^2\alpha}{\cos\alpha}; \\
    \left(\vec\omega + \vec\Omega\right) \times \overline{OQ}
        = \overline{0}
    \then \omega_y = -\Omega_y,\quad
        \omega_x = -\frac{\omega_y}{\tan\alpha}.
\end{gather}
Верно, результирующая угловая скорость направлена
вдоль~$OQ$ (см., например, 4.34):
\begin{equation}
    \omega_\Sigma = \frac{\Omega}{\tan\alpha} \hat x
        = \frac{v}{R\cos\alpha} \hat x.
\end{equation}

Кинетическая энергия вращения (в~главных осях):
\begin{equation}
    T_r = \frac{1}{2}
        \cvec{\omega_\Sigma\sin\alpha}{0}{\omega_\Sigma\cos\alpha}^T
        \begin{pmatrix}
            \frac{1}{4} mR^2 & 0 & 0 \\
            0 & \frac{1}{4} mR^2 & 0 \\
            0 & 0 & \frac{1}{2} mR^2
        \end{pmatrix}
        \cvec{\omega_\Sigma\sin\alpha}{0}{\omega_\Sigma\cos\alpha}
        = \left(\frac{1}{8} \sin^2\alpha 
        + \frac{1}{4} \cos^2\alpha\right) \omega_\Sigma^2 mR^2.
\end{equation}

Окончательно запишем
\begin{equation}
    T = T_r + \frac{1}{2} mv^2
        = \left(\frac{1}{8} \sin^2\alpha +
            \frac{1}{4} \cos^2\alpha\right)
            m \frac{v^2}{\cos^2\alpha} + \frac{1}{2} mv^2
        = mv^2 \left(\frac{3}{4} + \frac{1}{8}\tan^2\alpha\right).
\end{equation}

\textbf{Ответ:} \qquad
$T = \dfrac{mv^2}{8} \left(6 + \tan^2\alpha\right)$.
\hfill $\blacktriangleleft$

\textsl{Примечание.} В~настоящем решении~$\alpha = 0$
при~<<вертикальном>> расположении диска.

\end{document}
