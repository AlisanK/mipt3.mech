\documentclass{weekly}
\begin{document}
\maketitlew{Аналитическая механика}{1}{2}{5}

\paragraph{3.6.} Плоская фигура движется в~своей плоскости.
Найти положение точки~$A$, если известны скорость этой точки~$\vec v_A$,
скорость некоторой другой точки~$\vec v_0$ и~угловая скорость
фигуры~$\vec\omega$. Использовать полученный результат для~определения
положения мгновенного центра скоростей~$P$.

$\blacktriangleright$ По~теореме о~распределении скоростей
в~твёрдом теле
\begin{equation}\label{3.6:vA}
    \vec v_A = \vec v_0 + \vec\omega \times \overline{OA}.
\end{equation}
Умножим векторно уравнение~\eqref{3.6:vA} на~$\vec\omega$ слева:
\begin{gather}
    \vec\omega \times \left(\vec v_A - \vec v_0\right) =
            \vec\omega \times \vec\omega \times \overline{OA}
        = \vec\omega \cancelto{0}
            {\left(\vec\omega, \overline{OA}\right)}\quad -
            \overline{OA} \left(\vec\omega, \vec\omega\right)
        = -\omega^2 \cdot \overline{OA}; \\[\parskip]
    \overline{OA} = \frac{\vec\omega \times
            \left(\vec v_0 - \vec v_A\right)}{\omega^2}.
\end{gather}
В~частности, для~мгновенного центра скоростей~$P$ можно записать
$\vec v_P = \vec 0$.\\[-\parskip]

\textbf{Ответ:}\quad $\overline{OA} = \dfrac{\vec\omega \times
\left(\vec v_0 - \vec v_A\right)}{\omega^2}$;\qquad
$\overline{OP} = \dfrac{\vec\omega \times \vec v_0}{\omega^2}$.
\hfill $\blacktriangleleft$

\end{document}
